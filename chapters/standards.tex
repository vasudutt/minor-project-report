\chapter{Standards Adopted}

This chapter discusses the standards adopted during the development of the full stack application, including design standards, coding standards, and testing standards.

\section{Design Standards}
In the field of software engineering, adhering to design standards is crucial for creating robust and maintainable applications. While there are various design standards available, the following recommended practices were followed for the project:

\begin{itemize}
	\item \textbf{Use of UML Diagrams: } Unified Modeling Language (UML) diagrams were utilized to visually represent the system architecture, data models, and relationships between different components.
	\item \textbf{Consistent User Interface Design: } The user interface design followed industry best practices and aimed to provide a seamless and intuitive user experience. Consistency in terms of layout, color schemes, and typography was maintained to ensure familiarity and usability for users.
	\item \textbf{Database Design Standards: } The database schema design followed standard practices to ensure data integrity, normalization, and efficient querying. The structure of the database entities and relationships between them were carefully defined based on the requirements of the application.
\end{itemize}

\section{Coding Standards}
Coding standards play a vital role in producing clean, readable, and maintainable code. During the development of the full stack application, the following coding standards and best practices were adhered to:

\begin{enumerate}
	\item \textbf{ESLint Integration: } ESLint, a popular static code analysis tool, was integrated into the development workflow. ESLint helped enforce coding standards, detect potential errors, and ensure consistency throughout the codebase. Custom ESLint rules were configured to align with industry best practices.
	\item \textbf{TypeScript Usage: } TypeScript, a statically typed superset of JavaScript, was utilized in the project. TypeScript's strict type-checking capabilities helped catch errors early and improve code reliability. It enforced type annotations, making the codebase more robust and maintainable.
	\item \textbf{Naming Conventions: } Appropriate naming conventions were followed to ensure clarity and readability of the code. Descriptive and meaningful names were used for variables, functions, and components, following standard naming conventions such as camel case or Pascal case.
	\item \textbf{Code Organization: } The codebase was structured in a modular and organized manner. Functions and classes were kept concise and focused on a single task, promoting reusability and maintainability. Indentation and formatting were applied consistently to improve code readability.
\end{enumerate}

\section{Testing Standards}
While there was no explicit testing carried out in this project, it is important to acknowledge that there are established standards for quality assurance and testing in software development. Common standards followed for testing and verification include ISO/IEC/IEEE 29119, which provides guidelines for test processes, test documentation, and test techniques. While these specific standards were not implemented in this project, it is recognized that testing and verification are integral components of software development projects.

By adhering to design standards, utilizing tools like ESLint and TypeScript to enforce coding standards, and recognizing the importance of testing standards, the project aimed to maintain high-quality software development practices. These standards contribute to the overall quality, reliability, and maintainability of the full stack application.