\chapter{Problem Statement}

The problem statement for this project revolves around the limitations and gaps present in current available solutions for online property rentals. While platforms like Airbnb have successfully disrupted the market and provided users with a convenient way to find and rent accommodations, there are still areas that can be improved upon. The goal of this project is to develop a full stack application that addresses these limitations and provides a functional clone of Airbnb with enhanced features and user experience.

The following subsections present the Software Requirements Specification (SRS) according to the IEEE format, outlining the project planning, analysis, and system design.

\section{Project Planning}
The planning phase of the project involves defining the steps to be followed for successful execution of the development process. The following list of requirements and features outline the key aspects to be developed in the full stack application:

\begin{enumerate}
	\item \textbf{User Registration and Authentication:} Users should be able to register and authenticate using email and password credentials or through their existing Google or GitHub accounts.
	\item \textbf{Property Listing:} Users should be able to list their properties for rent, providing details such as title, description, images, location, room count, bathroom count, and guest count.
	\item \textbf{Property Search and Filtering:} Users should be able to search for properties based on location, room count, bathroom count, guest count, and other criteria. Advanced filtering options should be provided to refine search results.
	\item \textbf{Property Reservation:} Users should be able to make reservations for properties, specifying the start and end dates of their stay.
	\item \textbf{Favorite Properties: } Users should be able to add properties to their list of favorites.
	\item \textbf{User Profile Management:} Users should have the ability to manage their profiles, including updating their own listings, modifying the favorites list and viewing their past reservations.
	\item \textbf{Responsive UI/UX:} The user interface should be designed to be responsive, ensuring optimal viewing and interaction across different devices and screen sizes.
\end{enumerate}

\section{Project Analysis}
After collecting the requirements and conceptualizing the problem statement, a thorough analysis needs to be performed to identify any ambiguities, mistakes, or inconsistencies. This analysis phase ensures a clear understanding of the project scope and requirements, minimizing potential risks and challenges during the development process.

\section{System Design}
\subsection{Design Constraints}
The design constraints for the full stack application involve the utilization of specific technologies and frameworks to ensure compatibility and optimal performance. The following constraints are considered:
\begin{enumerate}
	\item \textbf{NextJS: } The application should be built using NextJS, leveraging its server-side rendering and client-side rendering capabilities.
	\item \textbf{Typescript:} Typescript should be used to enforce static typing and improve code quality.
	\item \textbf{Prisma:} The application should utilize Prisma as the ORM for database connectivity and management.
	\item \textbf{TailwindCSS:} The UI should be designed using TailwindCSS, taking advantage of its utility-first approach for rapid development.
	\item \textbf{MongoDB:} MongoDB should be used as the database for efficient storage and retrieval of property and user-related data.
	\item \textbf{Cloudinary:} Cloudinary should be integrated to handle media file storage and delivery.
\end{enumerate}

\subsection{System Architecture}
The system architecture for the full stack application involves the integration of various components and technologies. The architecture should follow a modular and scalable approach, allowing for easy maintenance and future enhancements. Key components of the system architecture include:

\begin{enumerate}
	\item \textbf{NextJS Framework:} The core framework for server-side rendering, client-side rendering, and routing.
	\item \textbf{Prisma ORM:} Facilitates database connectivity, query building, and data modeling.
	\item \textbf{MongoDB Database:} Stores property and user-related data in a flexible and scalable manner.
	\item \textbf{TailwindCSS:} Enables rapid UI development through utility classes and responsive design.
	\item \textbf{Cloudinary:} Manages storage and delivery of media files, optimizing performance.
\end{enumerate}

The deployment of the full stack application was done on Vercel, a cloud platform that specializes in hosting static sites and serverless functions. Vercel simplifies the deployment process and ensures scalability, reliability, and efficient updates through its robust continuous integration and continuous deployment (CI/CD) pipelines.

By adopting this system architecture, the full stack application is designed to be modular, scalable, and optimized for performance. The integration of NextJS, Prisma, MongoDB, TailwindCSS, and Cloudinary provides a comprehensive and efficient solution for property listing, search, reservation, and user profile management.

Through proper planning, analysis, and system design, the development process of the full stack application is well-structured and lays a solid foundation for successful implementation. The subsequent sections of this report will delve into the implementation details, showcasing how the project's objectives and requirements are translated into a fully functional and user-friendly application.